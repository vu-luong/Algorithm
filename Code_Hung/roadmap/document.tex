\documentclass[oneside,12pt]{book}
\usepackage[bottom=1.0in, top=1.0in, width=6.15in]{geometry}
\usepackage[pdftex]{graphicx}
\usepackage{amsmath}
\usepackage{amssymb}
\usepackage{tipa}
%\usepackage{txfonts}
\usepackage{textcomp}
%\usepackage{amsthm}
%\usepackage{array}
%\usepackage{xy}
\usepackage{fancyhdr}
\usepackage{hyperref}
\usepackage{lmodern}
\usepackage{standalone}%
\usepackage{titlesec}
\usepackage{tocloft}
\usepackage{lipsum} 
\usepackage{xcolor} 
\usepackage[utf8]{vietnam}
%\usepackage[utf8x]{inputenc}
\usepackage[vietnamese]{babel}
\usepackage{enumitem}
\usepackage{multicol}
\usepackage{multirow}
\usepackage{array}
\usepackage{mathrsfs}
\usepackage{setspace}
\hypersetup{
	colorlinks=true,
	linkcolor=black,
	filecolor=black,      
	urlcolor=black,
}


\pagestyle{fancy}
%\renewcommand{\chaptermark}[1]{\markboth{#1}{}}
\renewcommand{\sectionmark}[1]{\markright{\thesection\ #1}}
\fancyhf{}
\fancyhead[L]{\bfseries\thepage}
\fancyhead[R]{%
	% We want italics
	\itshape
	% The chapter number only if it's greater than 0
	\ifnum\value{chapter}>3 \fi
	% The chapter title
	%\leftmark
	}
%\renewcommand{\headrulewidth}{0.5pt}
\renewcommand{\footrulewidth}{0pt}
%\addtolength{\headheight}{0.5pt}
\setlength{\footskip}{0in}
\renewcommand{\footruleskip}{0pt}
\fancypagestyle{plain}{%
%\fancyhead{}
%\renewcommand{\headrulewidth}{0pt}
}

\renewcommand\thechapter{\Roman{chapter}}
\renewcommand\thesection{\arabic{section}.}
\renewcommand\thesubsection{\arabic{section}.\arabic{subsection}.}
%\titleformat{\chapter}{\LARGE\bfseries}{Chapter \thechapter \linebreak}{1em}{}

\renewcommand{\theequation}{\arabic{equation}}
\def\*#1{\mathbf{#1}}
\renewcommand{\thefigure}{\arabic{figure}}
\setstretch{1.5}

\titleformat{\chapter}[display]
{\normalfont\bfseries\LARGE}
{\chaptertitlename~\thechapter}{1pc}
{{\color{brown}\titlerule[1.5pt]}\vspace{1pc}}
\titleformat{name=\chapter,numberless}[display]
{\normalfont\bfseries\LARGE}{}{1pc}
{}

%
%\parindent 0in
%\parskip 0.05in
%
\addtolength{\cftchapnumwidth}{20pt}

%\setcounter{chapter}{1}

\begin{document}
\frontmatter



\tableofcontents
\clearpage
\ifodd\value{page}\else
\thispagestyle{empty}
\fi
%
\mainmatter
%

%\chapter{Các thuật toán phân cụm theo thứ bậc}
{\noindent\LARGE\normalfont\bfseries Lộ trình học trong hè\\ \\ }




\section{Nhập - xuất}
\textbf{Thời gian: }21 - 30/06/17\\
\textbf{Bài tập: }Bí kíp luyện rồng từ 20 - 32

\section{Câu lệnh rẽ nhánh - điều kiện}
\textbf{Thời gian: } 01 - 07/07/17\\
\textbf{Bài tập: } Bí kíp luyện rồng 101 - 125

\section{Vòng lặp}
\textbf{Thời gian: } 08 - 14/07/17\\
\textbf{Bài tập: } Bí kíp luyện rồng 231 - 260

\section{Xâu}
\textbf{Thời gian: } 15 - 21/07/17\\
\textbf{Bài tập: } Bí kíp luyện rồng 412, 413, 414, 420, 441, 442, 443, 447, 451, 452, 454, 456, 458, 459, 460.

\section{Mảng}
\textbf{Thời gian: } 22 - 28/07/17\\
\textbf{Bài tập: } Bí kíp luyện rồng 601 - 625

\section{Đệ quy}
\textbf{Thời gian: } 29 - 15/08/17\\
\textbf{Bài tập: } Bí kíp luyện rồng 801 - 839.

\textbf{Chú ý: } Phần này nâng cao hơn các phần trước, nên đọc thêm sách \textit{Giải thuật và lập trình} kết hợp tìm kiếm trên mạng để hiểu sâu hơn.

\section{Bài tập thêm}
Codeforces Round 400 đến Round 419: Làm các bài A.
Nếu làm xong các bài A thì làm tiếp các bài B.


\end{document}
